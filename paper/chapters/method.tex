\section{Method}
\subsection{Benchmarks}
The PRISM, Modest and Storm model checkers will be benchmarked using the Microban level set\footnote{Puzzles by David W. Skinner - \url{http://www.abelmartin.com/rj/sokobanJS/Skinner/David\%20W.\%20Skinner\%20-\%20Sokoban.htm}} by David W. Skinner. This set is part of the Large Test Suite and Open Test Suite\footnote{Test suite overview - \url{http://sokobano.de/wiki/index.php?title=Solver_Statistics}}, both often used to benchmark Sokoban solvers. The set contains 155 levels that can be solved quickly by specialised solvers. However, solving the stochastic variants using probabilistic model checkers is a much more computationally expensive task, as the entire state space has to be explored.

For each level and model checker, the same property is benchmarked: the maximum probability that the target state is reached. During the benchmark, the peak RAM usage is recorded, and the total time until the answer is calculated is measured. All levels are benchmarked twice: The first run is done with a $\mu$ value of 0.3, the second run is done with a $\mu$ value of 0.9. This is done to further identify differences in performance between the model checkers.

The PRISM model checker only supports models in the PRISM language. Therefore, it will be benchmarked using the generated models in the PRISM language. The Modest and Storm benchmarks are run on the generated models defined according to the JANI specification.

The benchmarks are run inside a virtual machine (VM) to ensure consistent results. This allows for an equal amount of resources to be allocated to the model checkers. The VM runs on Ubuntu 64-bit 20.04.4 LTS, and has access to four cores running at 4.0 GHz. Each model checker has an allocated 8 GB of RAM. This memory limit is assigned to the process itself, not the VM.

To prevent the benchmarks from running for an extensive amount of time, a 5-minute timeout is used for each level. Failing to find the maximum probability of reaching the target state within these 5 minutes means the benchmark has failed.

\subsection{Analysis}
\label{sec:analysis}
To better understand the implications that the stochastic additions have on the solutions generated by the model checkers, the following properties are analysed:
\begin{enumerate}
    \item What is the maximum probability of the target state being reached?
    \item What is the maximum probability of the optimal solution being reached?
    \item What is the expected number of moves in which the target state will be reached?
\end{enumerate}

The probability that a player makes the best move, $\mu$, is varied so that the effect of the stochastic additions can be made clear. The analysis is done on the entirety of the Microban level set, however, a model is skipped if the computation time for the analysis exceeds ten minutes. The probabilities are computed per level and then averaged to create results that are representative of the entire level set.