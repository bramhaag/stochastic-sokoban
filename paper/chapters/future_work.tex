\section{Future work}
The generated can be optimised further by reducing the bounds of the player position. Currently, it is bound between the lowest and highest reachable position index, but it is possible that many of these values in between the bounds are walls and thus not valid player positions. A solution to this would be to map all the positions on the board to a unique integer in the range $[0..n)$, with $n$ being the number of reachable tiles on the level. That way, all invalid positions are no longer considered, and the player's position will be bounded between $0$ and $n-1$.

In some cases, transitions are generated with guards that can never be satisfied. This can be a walk move towards an immovable box, or a push move where it is impossible for the player and box to be in those specific positions. Additionally, immovable boxes on a goal position are still considered when determining if a level is solved or not; these can also be optimised away. 

As described in \autoref{sec:rewards}, the model is modified to determine the expected number of moves. This influences the resulting amount of moves and thus is not optimal. This can be solved by using conditional reward properties to find the expected amount of moves for all solutions that reach the goal state. Baier et al. give an intricate solution for this \cite{Baier2017}. However, this method is not implemented in PRISM, Storm or Modest. An extension for PRISM is available that supports conditional properties, but rewards are not supported \cite{Mrcker2017}.